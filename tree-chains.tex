\documentclass{article}
\usepackage{amsmath}
\usepackage{graphics}
\usepackage{url}
\usepackage{mathtools}
\usepackage{tikz}
\usetikzlibrary{arrows,positioning}
\begin{document}
\title{Tree Chains}
\author{Peter Todd}
\date{FIXME}
\maketitle

\section{Introduction}

Linear blockchains offer a "one-sized-fits-all" solution to the
proof-of-publication problem. While this
offers maximum first-order security, the inherently poor $O(n^2)$ scalability
undermines the security of this simple design with second-order effects like
centralization.


\section{Refining proof-of-publication with TXO set commitments}

Let $B=(b_0,\dots,b_n)$ be a linear\footnote{Linear in the sense that $b_i$
follows $b_{i-1}$v; what form the cryptographic hash commitments take is
irrelevant.} set of blockheaders following Satoshi/Bitcoin proof-of-work
rules.\cite{bitcoin-paper} Each block contains a set of transactions
$T=(t_0,\dots,t_m)$. Each transaction in that set causes zero or more prior
transaction outputs to be marked as spent and creates zero or more new
transaction outputs.\footnote{A zero input, $>0$ output transaction can be used
to reward miners with new coins.} Each block \emph{commits} to the state of the
entire TXO set, that is all transaction outputs that have ever been created,
using an insertion-ordered sparse merkle
tree\cite{revocation-transparency-sparse-merkle-tree} or merkle mountain range.
A valid block may never commit to a TXO set that would cause a previously spent
transaction to be marked unspent.

When a transaction $t$ is published in a block $b_i$ we can say that the
blockchain acts as a \emph{witness} to the fact that the transaction was
published. The security of that witness is measured in expected work done:

\begin{equation}
    \omega(t) = \sum_{j=i}^n \operatorname{Work}(b_j)
\end{equation}

Assuming we do in fact have a jam-free publication mechanism\footnote{The
reader would do well to dwell on the "turtles all the way down" nature of the
fact that our proof-of-publication mechanism relies on a proof-of-publication
mechanism: a hopefully jam-free network.} for the block headers themselves
these headers are proof of the minimum amount of work required by an attacker
to rewrite history double-spend the transaction.

We will call $\omega(t)$ the measurement of the \emph{first-order} double-spend
security; implicit in that measurement is the assumption that all prior blocks
were valid. We would also like to know if the transaction output(s) the
transaction spent were themselves valid and unspent. For a given transaction
output $x$ created in block $b_i$ the blocks following $b_i$ act as a witness
to the fact that for output $x$ 


FIXME: so introduce first-order and second-order security here; we'll use the
concepts later when doing the cross-chain stuff.


\section{Proof-of-publication and merge-mining}

\begin{figure}
    \centering
    \include{figures/three-merge-mined-chains}
    \caption{Three merge mined chains}
    \label{fig:three-merge-mined-chains}
\end{figure}

Figure \ref{fig:three-merge-mined-chains} shows master blockchain $A$, and
two merge-mined blockchains $B$ and $C$. With regard to scalability the system as a whole may 


\cite{disentangling-crypto-coin-mining}


\section{The security/scalability tradeoff}




\bibliographystyle{plain}
\bibliography{tree-chains}

\end{document}
